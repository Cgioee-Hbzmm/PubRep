\documentclass[12pt]{article}
\usepackage[utf8]{inputenc}
\usepackage{emptypage}
\usepackage{amssymb}
\usepackage[a4paper]{geometry}
\usepackage{indentfirst}
\usepackage{amsthm}
\usepackage{amsmath}
\usepackage{physics}
\usepackage{amsfonts}
\usepackage{graphicx}
\usepackage{caption}
\usepackage{mathtools}
\usepackage{url}
\usepackage[nottoc]{tocbibind}
\usepackage{notoccite}
\usepackage{stmaryrd}
\usepackage[english]{babel}
\usepackage{witharrows}
\newtheorem{theorem}{Theorem}[section]
\newtheorem{corollary}{Corollary}[theorem]
\newtheorem{lemma}[theorem]{Lemma}
\theoremstyle{definition}
\newtheorem{definition}{Definition}[section]
\theoremstyle{remark}
\newtheorem*{remark}{Remark}
\newtheorem{prop}{Proposition}
\newcommand{\C}{\mathcal{C}}
\newcommand{\D}{\mathcal{D}}
\newcommand{\tC}{$\Tilde{C}$}
\title{CRT Three Co-primes}
\author{Rajeet Ghosh}
\date{July 2023}

\begin{document}

\maketitle

\section{Number of common keys between two devices generated by CRT three co-primes}
Let us take three co-primes $p_1$, $p_2$ and $p_3$.
We take $C =(c_1,c_2,c_3)$ and $D=(d_1,d_2,d_3)$\\
We define the keyring of $C$ as $\C$ $=\{(c_1,j,k)|~j\in \mathbb{Z}_{p_2}, k \in \mathbb{Z}_{p_3} \}\cup \{(i,c_2,k)|~i\in \mathbb{Z}_{p_1}, k \in \mathbb{Z}_{p_3} \} \cup \{(i,j,c_3)|~i\in \mathbb{Z}_{p_1}, j \in \mathbb{Z}_{p_2} \}$\\
We similarly define the keyring of $D$ as $\D  =\{(d_1,j,k)|~j\in \mathbb{Z}_{p_2}, k \in \mathbb{Z}_{p_3} \}\cup \{(i,d_2,k)|~i\in \mathbb{Z}_{p_1}, k \in \mathbb{Z}_{p_3} \} \cup \{(i,j,d_3)|~i\in \mathbb{Z}_{p_1}, j \in \mathbb{Z}_{p_2} \}$
\\
We define $C_i$ as fixing the i-th coordinate as $c_i$ and varying the remaining coordinates, and use a similar definition for $D_i$\\So  $C_1:=\{(c_1,j,k)|~j\in \mathbb{Z}_{p_2}, k \in \mathbb{Z}_{p_3} \}\\
C_2:=\{(i,c_2,k)|~i\in \mathbb{Z}_{p_1}, k \in \mathbb{Z}_{p_3} \} \\C_3:= \{(i,j,c_3)|~i\in \mathbb{Z}_{p_1}, j \in \mathbb{Z}_{p_2} \}\\ D_1:=
\{(d_1,j,k)|~j\in \mathbb{Z}_{p_2}, k \in \mathbb{Z}_{p_3} \}\\D_2:= \{(i,d_2,k)|~i\in \mathbb{Z}_{p_1}, k \in \mathbb{Z}_{p_3} \} \\D_3:= \{(i,j,d_3)|~i\in \mathbb{Z}_{p_1}, j \in \mathbb{Z}_{p_2} \}$
\\So $\mathcal{C}=\displaystyle\bigcup_{i=1}^{3} C_i$ and similarly $\mathcal{D}=\displaystyle\bigcup_{i=1}^{3} D_i$
\subsection{Case I: All entries are distinct $(c_1 \neq d_1, c_2 \neq d_2, c_3 \neq d_3$)}
\begin{lemma} \label{le2}
    The number of common elements of two keyrings, all of whose entries are different, is $2(p_1+p_2+p_3)- 6$.
    
\end{lemma}
\begin{proof}
    Since $c_i\neq d_i$ for all $1\leq i\leq 3, C_i\cap D_i = \phi $ Thus we only need to focus on $C_i\cap D_j$ for all $i\neq j$. Now,
    \begin{align*}
        \mathcal{C}\cap\mathcal{D}&=\Bigl(\displaystyle\bigcup_{i=1}^{3}C_i \Bigr)\cap\Bigl(\displaystyle\bigcup_{i=1}^{3}D_i\Bigl)
        \\&=\displaystyle\bigcup_{i=1}^{3}C_i\cap\Bigl(\displaystyle\bigcup_{\substack{j=1\\ j\neq i}}^{3}D_j\Bigl) =\displaystyle\bigcup_{i=1}^{3}\displaystyle\bigcup_{\substack{j=1\\ j\neq i}}^{3} (C_i\cap D_j)  \\
        &= (C_1 \cap D_2) \cup(C_1 \cap D_3) \cup(C_2 \cap D_1) \cup(C_2 \cap D_3) \cup(C_3 \cap D_1) \cup(C_3 \cap D_2)
    \end{align*}
    Therefore, we will apply the Inclusion-Exclusion Principle to estimate the number of elements of $\mathcal{C}\cap \mathcal{D}$. Before we do so, note that $(C_i\cap D_j) \cap (C_{i'}\cap D_{j'}) = \phi$ for all $i\neq i'$ and $j\neq j'$ [ As $C_i\cap C_{i'} = \phi$ and $D_j \cap D_{j'}=\phi$]. Also $(C_i\cap D_j) \cap (C_{i}\cap D_{j'})= (C_i\cap D_j\cap D_{j'})$  and  $(C_i\cap D_j) \cap (C_{i'}\cap D_{j})= (C_i\cap C_{i'}\cap D_j)$ \\Now $(C_1\cap D_2) = \{(c_1,d_2,k)| k \in \mathbb{Z}_{p_3}\}$ and so $|C_1 \cap D_2|=p_3$. Similarly, for $|C_i \cap D_j|=p_{k'}$ where $k'$ denotes the non-fixed entry of the set\\
    Additionally, $C_1 \cap D_2 \cap D_3 = (c_1, d_2, d_3)$ and $|C_1 \cap D_2 \cap D_3 |= 1$. In a similar fashion, $|(C_i \cap C_{i'} \cap D_j)|$ and $|C_i \cap D_j \cap D_j'|$ both equal $1$ Additionally, keep in mind that the sets $(C_i \cap C_{i'}\cap D_j)$ and $(C_i'\cap D_j\cap D_{j'})$ are singletons and represent different points. Consequently, their intersections with themselves and with one another produce an empty set.  

    Therefore 
        \begin{align*}
        |\mathcal{C}\cap\mathcal{D}|&=\displaystyle\sum_{i=1}^{3}\displaystyle\sum_{\substack{j\in \{1,2,3\}\\j\neq i}}|(C_i\cap D_j)| -  \displaystyle\sum_{\substack{i,i',j \in \{1,2,3\}\\i,i',j~ \text{distinct}}}|(C_i \cap C_{i'} \cap D_j)|
        \\& - \displaystyle\sum_{\substack{i,j,j' \in \{1,2,3\}\\i,j,j'~ \text{distinct}}}|(C_i \cap D_{j} \cap D_{j'})| \\
        &=2(p_1+p_2+p_3) - 3 -3=2(p_1+p_2+p_3)-6
        \end{align*}
\end{proof}
\subsection{Case 2: One entry of the tuple is identical }
\begin{lemma}\label{le1}
    Suppose the $i$-th entry of the distinct codes is identical, while the $j$-th and $k$-th entries are distinct. Then the number of common elements in keyrings generated by both would be
        $$p_jp_k+2p_i-2$$
\end{lemma}
\begin{proof}
    Since the formula would be symmetric, we do the case where the first entry is identical, and the rest of the cases would be similar
    Now since the first entry is similar, $C_1=D_1=\Tilde{C}$.\\
    So $\mathcal{C}=\Tilde{C}\cup C_2\cup C_3$ and $\mathcal{D}=\Tilde{C}\cup D_2 \cup D_3$ and $C_i\cap D_i=\phi $ for all $i=2,3$
    Thus 
    \begin{align*}
        \mathcal{C}\cap\mathcal{D}&=(\Tilde{C}\cup C_2\cup C_3)\cap(\Tilde{C}\cup D_2 \cup D_3) 
        \\&=\Tilde{C} \cup (\Tilde{C} \cap D_2) \cup (\Tilde{C} \cap D_3) \cup (\Tilde{C} \cap \ C_2) \cup (\Tilde{C} \cap C_3) \cup  (C_2 \cap D_3) \cup (C_3 \cap D_2) 
        \\&= \Tilde{C} \cup (C_2 \cap D_3) \cup (C_3 \cap D_2) \\&~~~~~~~~~~~~~~~~~~~~~~~~~~~~~~~~~~~~~~~~~~~~~~~~~~~~~~~~~~~~~\text{[ since $A \cap B \subset A,B$ for any set $A,B$ ]} 
    \end{align*}
    So using the inclusion-exclusion principle, we get\begin{align*}
        |\C\cap\D|&=|\Tilde{C}| + |C_2 \cap D_3| + |C_3 \cap D_2| -|\Tilde{C} \cap C_2 \cap D_3| - |\Tilde{C} \cap C_3 \cap D_2|   \\&-|C_2 \cap D_3 \cap C_3 \cap D_2| + |\Tilde{C}\cap C_2 \cap D_3 \cap C_3 \cap D_2| 
        \\&= |\Tilde{C}| + |C_2 \cap D_3| + |C_3 \cap D_2| -|\Tilde{C} \cap C_2 \cap D_3| - |\Tilde{C} \cap C_3 \cap D_2| 
        \\& ~~~~~~~~~~~~~~~~~~~~~~~~~~~~~~~~~~~~~~~~~~~~~~\text{[Since $C_i$ and $D_i$ are disjoint for the same index i]}
    \end{align*}
    Now to calculate each term, $|\tilde{C}|=|C_1|=|\{(c_1,j,k)|j \in \mathbb{Z}_{p_2}, k \in \mathbb{Z}_{p_3}\}|=p_2p_3$
    Using the same technique as above, we get $|C_2 \cap D_3|=|C_3 \cap D_2| =p_1 $ and $|\Tilde{C} \cap C_2 \cap D_3| = |\Tilde{C} \cap C_3 \cap D_2|=1$
    Therefore,
    \begin{align*}
        |\C\cap\D|=p_2p_3+2p_1-2
    \end{align*}
\end{proof}
\subsection{Case 3: Two entries are identical}
\begin{lemma}
    Suppose the $i$-th entry of the distinct codes is distinct, while the $j$-th and $k$-th entries are identical. Then the number of common elements in key rings generated by both would be
    
       $$p_i(p_j+p_k-1)$$
    \end{lemma}
\begin{proof}
    Again similar to the proof of Lemma 1.2, we assume that the first two entries are identical and the third entry is distinct. We define $ \hat{C} = C_1 \cup C_2 = D_1 \cup D_2$ . Therefore $\C=\hat{C} \cup \C_3 $ and $\D=\hat{C} \cup D_3$ where $C_3\cap D_3 = \phi$
    . Now,\\
        $\C \cap \D = (\hat{C} \cup C_3) \cap (\hat{C} \cup D_3)
       = \hat{C} \cup (\hat{C} \cap C_3) \cup (\hat{C} \cap D_3) \cup (C_3 \cap D_3)
       =\hat{C}$ \\
       Thus $|\C\cap\D|=|\hat{C}|$
    Now to calculate $|\hat{C}|$, we again apply the inclusion-exclusion principle to get,
    \begin{align*}
        |\hat{C}|&=|C_1\cup C_2|
                =|C_1| + |C_2| - |C_1 \cap C_2|\\
                &=p_2p_3 + p_1p_3 -p_3 &&[\text{From the proofs of Lemmae \ref{le2} and \ref{le1}} ]
    \end{align*}
\end{proof}
\end{document}
